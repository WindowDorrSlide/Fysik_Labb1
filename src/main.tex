\documentclass[11p, titlepage, oneside, a4paper]{article}
% Packages
\usepackage{amsmath}
\usepackage{graphicx}
\usepackage{hyperref}
\usepackage[english,swedish]{babel}
\usepackage[
    backend=biber,
    style=authoryear-ibid,
    sorting=ynt
]{biblatex}
\usepackage[utf8]{inputenc}
\usepackage[T1]{fontenc}
%Källor
\addbibresource{mall.bib}
\graphicspath{ {./images/} }

% Ändra de rader som behöver ändras
\def\inst{Teknikprogrammet}
\def\typeofdoc{Laborationsrapport}
\def\course{Fysik-1 150p}
\def\pretitle{Laboration 1}
\def\title{Rörelse: Hastighet och acceleration}
\def\name{Magnus Silverdal}
\def\username{magnus.silverdal}
\def\email{\username{}@ga.ntig.se}
\def\graders{Magnus Silverdal}

\begin{document}

\begin{titlepage}
		\thispagestyle{empty}
		\begin{large}
			\begin{tabular}{@{}p{\textwidth}@{}}
				\textbf{NTI gymnasiet \hfill \today} \\
				\textbf{\inst} \\
				\textbf{\typeofdoc} \\
			\end{tabular}
		\end{large}
		\vspace{10mm}
		\begin{center}
			\LARGE{\pretitle} \\
			\huge{\textbf{\course}}\\
			\vspace{10mm}
			\LARGE{\title} \\
			\vspace{15mm}
			\begin{large}
				\begin{tabular}{ll}
					\textbf{Namn} & \name \\
					\textbf{E-mail} & \texttt{\email} \\
				\end{tabular}
			\end{large}
			\vfill
            \includegraphics[width=0.5\textwidth]{images/NTI Gymnasiet_Symbol_print_svart}
			\vfill
            \large{\textbf{Handledare}}\\
			\mbox{\large{\graders}}
		\end{center}
	\end{titlepage}

    \begin{otherlanguage}{english}
	\begin{abstract}
        This labb examines the change over velocity and acceleration of a wagon rolling down a slope.
        The experiment shows that the wagons velocity changes at a constant rate making the acceleration a constant linear line.
    \end{abstract}
    \end{otherlanguage}
    % Om arbetet är långt har det en innehållsförteckning, annars kan den utelämnas
	\pagenumbering{roman}
	\tableofcontents
	
	% och lägger in en sidbrytning
	\newpage

	\pagenumbering{arabic}
	
	% i Sverige har vi normalt inget indrag vid nytt stycke
	\setlength{\parindent}{0pt}
	% men däremot lite mellanrum
	\setlength{\parskip}{10pt}
	
	\section{Syfte och frågeställning}
		Syftet med laborationen är att analysera rörelse för en vagn som rullar längs en bana och beräkna hastighet och
        acceleration under rörelsen.

	\section{Bakgrund och teori}
        Med utgångspunkt från en film av förloppet kan mjukvara för motion tracking utnyttjas för att få fram vagnens
        position vid olika tidpunkter.
        Denna information används sedan tillsammans med definitionerna av medelhastighet $v_m = \frac{\Delta s}{\Delta t}$ och medelacceleration $a_m = \frac{\Delta v}{\Delta t}$ för att beräkna ett approximativt värde
        för hastigheten och accelerationen som funktion av tiden.
        Med ett tillräckligt kort tidssteg så blir medelvärdet ungefär lika med momentanvärdet och i filmen är tidssteget som störst $\frac{1}{25}$ sekund.  \parencite{impuls}
	

	\section{Metod och materiel}
        \begin{enumerate}
            \item Vagn
            \item Lutande plan med ställning
            \item Linjal
            \item Kamera
            \item Dator
        \end{enumerate}
        
        Det lutande planet monteras på ställningen så att den ena änden är rund 5 cm över bordet, se figur\ref{fig:lutandeplan}.
        Linjalen placeras längs planet så att det finns en längdskala  i filmen.
        Kameran placeras stadigt, mittemot uppställningen, på ett avstånd så att hela rörelsen ryms i filmen utan att
        kameran behöver flyttas.
        Vagnen rullas sedan nedför planet samtidigt som kameran filmar rörelsen.
        Försöket placeras så att ljusförhållanden och bakgrund ger en så tydlig och skarp film som möjligt.
        
        \begin{figure}[!h]
            \centering
            \includegraphics[width=0.8\textwidth]{images/lutandePlan}
            \caption{Uppbyggnad av lutande plan}
            \label{fig:lutandeplan}
        \end{figure}

	\section{Analys och beräkning}
    Filmen analyserades sedan med mjukvaran Tracker för att få fram en tabell med positionen som funktion av
    tiden.
    Denna data presenteras i tabell\ref{table:result}.
    Tabellen illustreras i Figur\ref{fig:stracka} med sträckan som funktion av tiden.

    \begin{figure}[!h]
        \centering
        \includegraphics[width=0.8\textwidth]{images/sträcka}
        \caption{Distans över Tid tabell i Excel}
        \label{fig:stracka}
    \end{figure}
        

    Datan importeras i Excel och hastigheten beräknas med hjälp av formeln
    \begin{equation}
        v_m = \frac{\Delta s}{\Delta t}
    \label{eq:medelhastighet}
    \end{equation}
    
    \section{Slutsats och resultat} 
        Resultatet av beräkningarna illustreras i Figurerna\ref{fig:hastighet} och\ref{fig:acceleration}.
Hastigheten är linjärt och vagnen accelererar med circa $0.1m/s^2$.

        \begin{figure}[!h]
            \centering
            \includegraphics[width=0.8\textwidth]{images/hastighet}
            \caption{Hastighet över Tid tabell i Excel}
            \label{fig:hastighet}
        \end{figure}

        \begin{figure}[!h]
            \centering
            \includegraphics[width=0.8\textwidth]{images/acceleration}
            \caption{Acceleration över Tid tabell i Excel}
            \label{fig:acceleration}
        \end{figure}


    \section{Diskussion}
        Resultatet som presenteras kan anses trovärdig även om accelerationen och hastigheten variera mycket. En
        anledning till detta kan vara kamerans hastighet.
        När vagnen rör sig ner för planet kan det vara att den mellan två bilder ha rört sig lite mer än mellan två
        andra.
        Denna liten skillnad i distansen amplifieras sedan vid beräkningen av hastigheten samt acceleration.
        I figur\ref{fig:distanshastighet} syns att ojämnheterna i distansen ligger vid samma ställe där hastigheten variera.
        I samma figur syns även en trendlinje för både Distansen och Hastigheten.
        Båda trendlinjer är parallela med varandra vilket även validerar att hastighetens funktion kan anses vara riktig då hastigheten är en funktion av distansen över tid.


        \begin{figure}[!h]
            \centering
            \includegraphics[width=0.8\textwidth]{images/DistansHastighet}
            \caption{Distans och Hastighet i Excel}
            \label{fig:distanshastighet}
        \end{figure}






    \begin{table}
            \begin{center}
            \begin{tabular}{ |c|c| }
            \hline
            Position (m) & Tid (s)  \\
            \hline
                0.00 & 0.00 \\
                1.10E+00 & 3.89E-04 \\
                1.13E+00 & 7.41E-04 \\
                1.16E+00 & 2.40E-03 \\
                1.19E+00 & 4.27E-03 \\
                1.23E+00 & 6.35E-03 \\
                1.26E+00 & 8.56E-03 \\
                1.31E+00 & 1.11E-02 \\
                1.34E+00 & 1.39E-02 \\
                1.37E+00 & 1.67E-02 \\
                1.40E+00 & 1.99E-02 \\
                1.43E+00 & 2.32E-02 \\
                1.47E+00 & 2.68E-02 \\
                1.50E+00 & 3.11E-02 \\
                1.53E+00 & 3.57E-02 \\
                1.56E+00 & 4.08E-02 \\
                1.59E+00 & 4.64E-02 \\
                1.64E+00 & 5.26E-02 \\
                1.67E+00 & 5.91E-02 \\
                1.71E+00 & 6.60E-02 \\
                1.74E+00 & 7.35E-02 \\
                1.77E+00 & 8.17E-02 \\
                1.80E+00 & 9.01E-02 \\
                1.83E+00 & 9.90E-02 \\
                1.87E+00 & 1.08E-01 \\
                1.90E+00 & 1.18E-01 \\
                1.93E+00 & 1.28E-01 \\
                1.96E+00 & 1.38E-01 \\
                2.01E+00 & 1.49E-01 \\
                2.04E+00 & 1.60E-01 \\
                2.07E+00 & 1.72E-01 \\
                2.11E+00 & 1.84E-01 \\
                2.14E+00 & 1.96E-01 \\
                2.17E+00 & 2.08E-01 \\
                2.20E+00 & 2.21E-01 \\
                2.23E+00 & 2.34E-01 \\
                2.27E+00 & 2.48E-01 \\
                2.30E+00 & 2.61E-01 \\
                2.35E+00 & 2.75E-01 \\
                2.38E+00 & 2.89E-01 \\
                2.41E+00 & 3.04E-01 \\
                2.44E+00 & 3.19E-01 \\
                2.47E+00 & 3.35E-01 \\
                2.51E+00 & 3.49E-01 \\
                2.54E+00 & 3.67E-01 \\
                2.57E+00 & 3.83E-01 \\
                2.60E+00 & 4.00E-01 \\
                2.63E+00 & 4.16E-01 \\
                2.67E+00 & 4.33E-01 \\
                2.71E+00 & 4.48E-01 \\
                2.75E+00 & 4.64E-01 \\
                2.78E+00 & 4.82E-01 \\
                2.81E+00 & 4.99E-01 \\
                2.84E+00 & 5.17E-01 \\
                2.87E+00 & 5.35E-01 \\
                2.91E+00 & 5.54E-01 \\
                2.94E+00 & 5.73E-01 \\
            \vdots & \vdots \\
            \hline
            \end{tabular}
            \caption{Mätvärden}
            \label{table:result}
            \end{center}
        \end{table}

    
    \printbibliography

\end{document}

